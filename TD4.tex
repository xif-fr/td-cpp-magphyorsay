\documentclass{book}
\setlength{\parindent}{0cm}
\usepackage[textwidth=17cm]{geometry}

\usepackage[utf8]{inputenc}
\usepackage[french]{babel}
\usepackage{amsmath,amssymb}

\usepackage{graphicx}
\usepackage{psfrag}
\usepackage{caption}
\usepackage{subcaption}
\usepackage{verbatim}

\usepackage{minted}		% Coloration syntaxique
\usepackage[T1]{fontenc}	% Style de ~ incorrect
\usepackage{lmodern}		% Style de ~ incorrect
%\usemintedstyle{upsud}
\newcommand{\inline}[1]{\mintinline[breaklines]{c++}{#1}}

% Meilleures couleurs
\usepackage{xcolor}
\definecolor{red}{RGB}{221,42,43}
\definecolor{green}{RGB}{132,184,24}
\definecolor{blue}{RGB}{0,72,112}
\definecolor{orange}{RGB}{192,128,64}
\definecolor{gray}{RGB}{107,108,110}

\usepackage[onehalfspacing]{setspace}
\setstretch{1.02}

% Solutions encadrées
\usepackage{tikz}
\usepackage[framemethod=tikz]{mdframed}
\newmdenv[
  singleextra={
    \fill[blue] (P) rectangle ([xshift=-15pt]P|-O);
    \node[overlay,anchor=south east,rotate=90,font=\color{white}] at (P) {\sf\textbf{Correction}};
  },
  firstextra={
    \fill[blue] (P) rectangle ([xshift=-15pt]P|-O);
    \node[overlay,anchor=south east,rotate=90,font=\color{white}] at (P) {\sf\textbf{Correction}};
  },
  secondextra={
    \fill[blue] (P) rectangle ([xshift=-15pt]P|-O);
    \node[overlay,anchor=south east,rotate=90,font=\color{white}] at (P) {\sf\textbf{Correction}};
  },
  backgroundcolor=blue!2,
  linecolor=blue,
  skipabove=12pt,
  skipbelow=12pt,
  innertopmargin=0.4em,
  innerbottommargin=0.4em,
  innerrightmargin=2.7em,
  rightmargin=0.7em,
  innerleftmargin=1.7em,
  leftmargin=0.7em,
]{correction}

% Pour cacher/montrer les solutions, décommenter/commenter les 3 lignes ci-dessous
\usepackage{comment}
% \renewenvironment{correction}{}{}
% \excludecomment{correction}

% Fancy chapters
\makeatletter
  \renewcommand{\@chapapp}{TD}
\makeatother

\usepackage{fancyhdr}
\usepackage{fncychap}
  \ChTitleVar{\Huge\bfseries\sffamily\color{blue}}
  \ChNameVar{\raggedleft\fontsize{22}{16}\selectfont\sffamily\color{blue}}
  \ChNumVar{\raggedleft\fontsize{60}{62}\selectfont\sffamily\color{blue}}

% Fancy sections
\usepackage{titlesec}
\titlespacing*{\chapter}{0pt}{-50pt}{40pt}
\titleformat{\section}[block]
  {\Large\bfseries\sffamily\color{blue}}
  {\thesection}
  {1em}
  {}

\newmdenv[nobreak,backgroundcolor=red!20,roundcorner=10pt,linecolor=white]{warning}

\newenvironment{prompt}{\begin{quote}\color{blue!75}\tt\$\,
}{\end{quote}}

\newcommand{\cc}{\mbox{C}}
\newcommand{\cpp}{\mbox{C\vspace{.5em}\protect\raisebox{.2ex}{\footnotesize++~}}}

\def\filename{\emph}

\newcommand{\ham}{\mathcal{H}}
\newcommand{\en}{\mathrm{E}}
\newcommand{\ket}[1]{\left\vert#1\right\rangle}
\newcommand{\bra}[1]{\left\langle#1\right\vert}
\newcommand{\ps}[2]{\left\langle#1\middle\vert#2\right\rangle}
\newcommand{\psop}[3]{\left\langle#1\middle\vert#2\middle\vert#3\right\rangle}
\newcommand{\com}[2]{\left[#1,#2\right]}
\newcommand{\avg}[1]{\left<#1\right>}
\newcommand{\abs}[1]{\left|#1\right|}


\newcommand{\diff}{\mathrm{d}}
\newcommand{\ud}{\,\mathrm{d}}
\newcommand{\Tr}{\mathrm{Tr}\,}
\newcommand{\tr}{\mathrm{tr}\,}
\newcommand{\im}{\mathrm{Im}\,}
\newcommand{\re}{\mathrm{Re}\,}
\newcommand{\nn}{\nonumber}
\newcommand{\be}{\begin{equation}}
\newcommand{\ee}{\end{equation}}
\newcommand{\ba}{\begin{eqnarray}}
\newcommand{\ea}{\end{eqnarray}}

\renewcommand{\l}{\left}
\renewcommand{\r}{\right}

\graphicspath{{Figures/}}

\usepackage{hyperref}

\begin{document}
\setcounter{chapter}{3}
\chapter{\label{ch:surcharge}Surcharge d'opérateurs}

Ce TD a pour but d'introduire la notion de surcharge d'opérateur afin de pouvoir écrire des expressions naturelles du type \inline{Complexe moinsUn = Complexe(0,1) * Complexe(0,1)}, et introduit l'utilisation du mot-clé \inline{this}.

\section{Surcharge d'opérateurs}
Surcharger les opérateurs binaires \inline{operator+}, \inline{operator-}, \inline{operator/}, \inline{operator*} et l'opérateur unaire \inline{Complexe operator-(Complexe)}. Il y a deux façons de faire :
\begin{enumerate}
\item Les opérateurs sont des fonctions et non des membres de la classe \texttt{Complexe}. Ils prennent donc tous leurs opérandes par argument. Exemple : \inline{Complexe operator+ (Complexe a, Complexe b)}. C'est la façon la plus intuitive. On pourra soit les déclarer à l'extérieur de la classe \inline{Complexe}, soit les déclarer comme fonctions amies\footnote{Rappel : Une fonction déclarée \inline{friend} n'est \emph{jamais} une fonction membre (même s'il faut la déclarer à l'intérieur de la classe), il ne font donc pas utiliser l'opérateur de résolution de portée \inline{Complexe::} lors de la définition. En revanche, une fonction \inline{friend} a accès aux membres \inline{public} et \inline{private} de la classe. Cela simplifie un peu l'implémentation.} (\inline{friend}) à l'intérieur de la classe \inline{Complexe}. On parle de \textit{non-member operator overloading}.
\item Les opérateurs prennant en premier argument un \texttt{Complexe} peuvent être des méthodes membres de la classe \texttt{Complexe}. Le premier opérande est implicitement l'objet lui même, ie. \inline{*this}. Exemple : \inline{Complexe Complexe::operator+ (Complexe b) const}. On parle de \textit{member operator overloading}.
\end{enumerate}
Ces deux façons sont équivalentes dans la plupart des cas. Une exception : si \texttt{a} n'est pas un \texttt{Complexe} mais est \emph{convertible en \texttt{Complexe}}, l'opération sera valide avec du \textit{non-member overloading}, alors qu'elle ne le sera pas avec du \textit{member overloading}. En général, il faut imaginer les cas d'utilisation et choisir la façon qui ne pose le moins de problèmes. Dans notre cas, il n'y a pas de réelle différence. \\

Surcharger également les opérateurs membres \inline{operator=},\inline{operator+=}, \inline{operator-=}, \inline{operator*=}, \inline{operator/=}. Ces opérateurs ont une signature du type ou \inline{Complexe& operator= (const Complexe&)}, et \emph{sont} des fonctions membres, il faut donc utiliser l'opérateur de résolution de portée \inline{Complexe::} lors de la définition. \\

On reprendra les exercices 1.7 / 2.4 (somme d'exponentielles) des premiers TDs afin de tester les opérateurs surchargés.

\begin{correction}

\subsection*{Avec des fonctions amies, \textit{non-member overloading}}

On modifie la classe \inline{Complexe} en rajoutant
\begin{minted}[breaklines]{c++}
class Complexe
{
public:
  ...

  // affectation de autre à soi
  Complexe& operator= (Complexe& autre);
  
  // addition de `a' et `b'
  friend Complexe operator+ (Complexe a, Complexe b);
  // addition de `b' à soi
  Complexe& operator+= (Complexe b);
  // opposé de `c'
  friend Complexe operator- (Complexe c);
  // soustraction de `a' et `b'
  friend Complexe operator- (Complexe a, Complexe b);
  // soustraction de `b' à soi
  Complexe& operator-= (Complexe b);
  // multiplication de `a' par `b' (que `a' et `b' soient scalaires ou complexes)
  friend Complexe operator* (Complexe a, Complexe b);
  friend Complexe operator* (Complexe a, double b);
  friend Complexe operator* (double a, Complexe a);
  // multiplication de `b' à soi
  Complexe& operator*= (Complexe b);
  Complexe& operator*= (double b);
  // division de `a' par `b' :
  friend Complexe operator/ (Complexe a, Complexe b);
  friend Complexe operator/ (Complexe a, double b);
  friend Complexe operator/ (double a, Complexe b);
  // division de soi par `b' :
  Complexe& operator/= (Complexe b);
  Complexe& operator/= (double b);

};
\end{minted}
L'utilisation du mot-clé \inline{friend} permet si nécessaire d'avoir accès aux membres privés de \inline{Complexe}. Dans le cas où les membres sont totalement encapsulés (ie. tous \inline{private}), l'utilisation de \inline{friend} permet un accès immédiat à ces membres, permettant d'éviter de multiples appels aux \emph{getters} et \emph{setters}. Il faut garder que les fonctions amis ne sont \emph{pas} des méthodes, bien qu'elles soient déclarées à l'intérieur de la classe.

\vspace{1em}

Pour toutes les méthodes ci-dessus, il est possible (et même recommandé pour éviter des copies inutiles) de prendre les objets \inline{Complexe} par référence constante :
\begin{minted}[breaklines]{c++}
class Complexe
{
public:
  ...
  friend Complexe operator+ (const Complexe&, const Complexe&);
  friend Complexe operator- (const Complexe&);
  ...
  Complexe& operator= (const Complexe&);
};
\end{minted}
Toutefois, par souci de clarté, on a gardé un passage par copie dans cette correction. Un objet de type \inline{Complexe} ne prennant que $2\times 8$ octets (contre $8$ pour un pointeur), la différence de performance est probablement peu perceptible, d'autant plus que le compilateur effectue certaines optimisations.

\vspace{1em}

L'implémentation est immédiate grâce au constructeur :

\begin{minted}[breaklines]{c++}
// Bien que déclaré dans la classe Complexe operator+ n'est pas un membre de Complexe, à cause du mot-clé friend
Complexe operator+ (Complexe a, Complexe b) {
  return Complexe(a.p_real + b.p_real, a.p_imag + b.p_imag);
}

// operator= est un membre de Complexe, donc il faut utiliser Complexe:: pour le définir
Complexe& Complexe::operator= (const Complexe& autre) {
  p_real = autre.p_real;
  p_imag = autre.p_imag;
  return *this;
}

// On peut utiliser les autres opérateurs déjà définis pour en définir d'autres
Complexe& Complexe::operator+= (const Complexe& autre) {
  *this = *this + autre;
  return *this;
}
\end{minted}

Notons que l'implémentation de l'opérateur d'affectation \inline{operator=} est triviale. On aurait très bien pu laisser le compilateur le définir par défaut (tout comme un constructeur par copie) :
\begin{minted}[breaklines]{c++}
class Complexe
{
public:
  ...
  Complexe (const Complexe&) = default;
  Complexe& operator= (const Complexe&) = default;
};
\end{minted}

\subsection*{Sans fonctions amies, \textit{non-member overloading}}

Ici, on choisit de ne pas faire de fonctions amies. On déclare donc de simples fonctions à l'extérieur de la classe :

\begin{minted}[breaklines]{c++}
class Complexe
{
public:
  ...
  // affectation de autre à soi
  Complexe& operator= (Complexe& autre);
  
  // addition de `b' à soi
  Complexe& operator+= (Complexe b);
  // soustraction de `b' à soi
  Complexe& operator-= (Complexe b);
  // multiplication de `b' à soi
  Complexe& operator*= (Complexe b);
  Complexe& operator*= (double b);
  // division de soi par `b' :
  Complexe& operator/= (Complexe b);
  Complexe& operator/= (double b);
  ...
};

// addition de `a' et `b'
Complexe operator+ (Complexe a, Complexe b);
// opposé de `c'
Complexe operator- (Complexe c);
// soustraction de `a' et `b'
Complexe operator- (Complexe a, Complexe b);
// multiplication de `a' par `b' (que `a' et `b' soient scalaires ou complexes)
Complexe operator* (Complexe a, Complexe b);
Complexe operator* (Complexe a, double b);
Complexe operator* (double a, Complexe a);
// division de `a' par `b' :
Complexe operator/ (Complexe a, Complexe b);
Complexe operator/ (Complexe a, double b);
Complexe operator/ (double a, Complexe b);
\end{minted}

L'implémentation est alors :

\begin{minted}[breaklines]{c++}
Complexe& Complexe::operator+= (const Complexe& autre) {
  p_real += autre.p_real;
  p_imag += autre.p_imag;
  return *this;
}

Complexe operator+ (Complexe a, Complexe b) {
  a += b;
  return a;
}
\end{minted}
Ici, on a pris l'argument \inline{a} par copie, on peut donc le modifier sans rirsque de modifier la variable de l'utilisateur !

Noter que l'on a renversé les rôles par rapport aux fonctions amies : c'est \inline{operator+} qui utilise \inline{operator+=} et non l'inverse. En effet, la fonction n'étant pas amie, elle n'a pas accès aux attributs, et il est alors plus simple d'utiliser la méthode \inline{operator+=}.

\subsection*{Une troisième façon de faire : \textit{member overloading}}

On peut aussi déclarer l'addition, la soustraction, la multiplication par un scalaire... comme méthodes constantes de la classe, comme ceci :

\begin{minted}[breaklines]{c++}
class Complexe
{
public:
  ...
  // affectation de autre à soi
  Complexe& operator= (Complexe& autre);
  
  // addition de `a' et `b'
  Complexe operator+ (Complexe b) const;
  // addition de `b' à soi
  Complexe& operator+= (Complexe b);
  // opposé de `c'
  Complexe operator- () const;
  // soustraction de `a' et `b'
  Complexe operator- (Complexe b) const;
  // soustraction de `b' à soi
  Complexe& operator-= (Complexe b);
  // multiplication de `a' par `b' (que `a' et `b' soient scalaires ou complexes)
  Complexe operator* (Complexe b) const;
  Complexe operator* (double b) const;
  // multiplication de `b' à soi
  Complexe& operator*= (Complexe b);
  Complexe& operator*= (double b);
  // division de `a' par `b' :
  Complexe operator/ (Complexe b) const;
  Complexe operator/ (double b) const;
  // division de soi par `b' :
  Complexe& operator/= (Complexe b);
  Complexe& operator/= (double b);

  ...
};

// multiplication de `a' par `b' (cas où `a' n'est pas un Complexe)
Complexe operator* (double a, Complexe b);
// division de `a' par `b' (cas où `a' n'est pas un Complexe)
Complexe operator/ (double a, Complexe b);
\end{minted}

La multiplication ou division d'un \inline{double} par un \inline{Complexe} ne peuvent être des méthodes de la classe \inline{Complexe} car elles "agissent sur un \inline{double}", pas un \inline{Complexe}. Autement dit, seules les fonctions dont le premier argument \texttt{a} était un \inline{Complexe} peuvent être transformées en méthodes constantes.

L'implémentation est alors :
\begin{minted}[breaklines]{c++}
Complexe& Complexe::operator+= (const Complexe& autre) {
  p_real += autre.p_real;
  p_imag += autre.p_imag;
  return *this;
}

// operator+ est ici un membre de Complexe, donc il faut utiliser Complexe:: pour le définir
Complexe Complexe::operator+ (Complexe b) const {
  b += *this;
  return b;
}
\end{minted}

Noter que l'on a renversé les rôles : c'est \inline{operator+} qui utilise \inline{operator+=} et non l'inverse. 

\end{correction}

%-----------------------------------------------------------

\section{L'opérateur de flux \mintinline{c++}{<<}}
L'opérateur de flux \inline{operator<<} peut également être surchargé pour pouvoir écrire
\begin{minted}[breaklines]{c++}
Complexe i(0,1);
cout << i << endl;
\end{minted}
La signature à utiliser est \inline{std::ostream& operator<< (std::ostream& out, const Complexe& nombre)}. En effet, \inline{std::cout} est un objet de type \inline{std::ostream} spécialement implémenté pour afficher dans la console\footnote{D'ailleurs, \inline{std::ofstream} étant aussi de type \inline{std::ostream}, cette surcharge de \inline{operator<<} fonctionnera aussi pour l'écriture de fichiers. Même chose pour \inline{std::ostringstream}.}. Une fois cet opérateur surchargé, on prendra soin de tester tous les opérateurs précédemment surchargés.

\begin{correction}
D'abord, assurons-nous que le type \inline{ostream} est déclaré, via
\begin{minted}[breaklines]{c++}
#include <ostream>
\end{minted}
Ensuite il faut déclarer l'opérateur de flux via
\begin{minted}[breaklines]{c++}
std::ostream& operator<< (std::ostream& out, const Complexe& nombre);
\end{minted}
Nous ne sommes pas obligés de déclarer cette fonction dans la classe \inline{Complexe} précédée du mot-clé \inline{friend} car pour afficher un nombre complexe dans la console nous pouvons nous contenter d'utiliser les membres \inline{public} de la classe. (De manière générale, quand on affiche un objet dans la console, on n'a pas besoin d'afficher des informations qui ne sont pas visibles par l'utilisateur, ie. \inline{private}.)

On notera l'utilisation de \inline{std::} car utiliser \inline{using namespace} à l'intérieur d'un fichier \filename{*.h} est une très mauvaise idée, et doit être évité, sinon, tous les fichiers qui incluent ce fichier \emph{*.h} subiront automatiquement l'usage de cet espace de nom.

Finalement, une possible implémentation de l'opérateur de flux peut-être
\begin{minted}[breaklines]{c++}
ostream& operator<< (ostream& out, const Complexe& nombre) {
  if (nombre.imag() >= 0) {
    out << nombre.real() << " + " << nombre.imag() << "i";
  } else {
    out << nombre.real() << " - " << -nombre.imag() << "i";
 }
 return out;
}
\end{minted}
Ici, nous n'avons plus besoin d'utiliser \inline{std::} car nous sommes dans un fichier source \filename{*.cpp} dans lequel on aura pris soin d'insérer \inline{using namespace std;}. Ce n'est pas un problème ici, car un fichier \filename{*.cpp} ne peut pas être inclus dans un autre, donc l'instruction \inline{using namespace std;} a de l'effet uniquement à l'intérieur de ce fichier.
\end{correction}



\end{document}
